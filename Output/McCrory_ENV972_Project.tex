% Options for packages loaded elsewhere
\PassOptionsToPackage{unicode}{hyperref}
\PassOptionsToPackage{hyphens}{url}
%
\documentclass[
  12pt,
]{article}
\usepackage{lmodern}
\usepackage{amssymb,amsmath}
\usepackage{ifxetex,ifluatex}
\ifnum 0\ifxetex 1\fi\ifluatex 1\fi=0 % if pdftex
  \usepackage[T1]{fontenc}
  \usepackage[utf8]{inputenc}
  \usepackage{textcomp} % provide euro and other symbols
\else % if luatex or xetex
  \usepackage{unicode-math}
  \defaultfontfeatures{Scale=MatchLowercase}
  \defaultfontfeatures[\rmfamily]{Ligatures=TeX,Scale=1}
  \setmainfont[]{Times New Roman}
\fi
% Use upquote if available, for straight quotes in verbatim environments
\IfFileExists{upquote.sty}{\usepackage{upquote}}{}
\IfFileExists{microtype.sty}{% use microtype if available
  \usepackage[]{microtype}
  \UseMicrotypeSet[protrusion]{basicmath} % disable protrusion for tt fonts
}{}
\usepackage{xcolor}
\IfFileExists{xurl.sty}{\usepackage{xurl}}{} % add URL line breaks if available
\IfFileExists{bookmark.sty}{\usepackage{bookmark}}{\usepackage{hyperref}}
\hypersetup{
  pdftitle={Long-term trends in cadmium and lead concentrations in Mytilus edulis samples from European Atlantic Coast},
  pdfauthor={Sena McCrory},
  hidelinks,
  pdfcreator={LaTeX via pandoc}}
\urlstyle{same} % disable monospaced font for URLs
\usepackage[margin=2.54cm]{geometry}
\usepackage{longtable,booktabs}
% Correct order of tables after \paragraph or \subparagraph
\usepackage{etoolbox}
\makeatletter
\patchcmd\longtable{\par}{\if@noskipsec\mbox{}\fi\par}{}{}
\makeatother
% Allow footnotes in longtable head/foot
\IfFileExists{footnotehyper.sty}{\usepackage{footnotehyper}}{\usepackage{footnote}}
\makesavenoteenv{longtable}
\usepackage{graphicx,grffile}
\makeatletter
\def\maxwidth{\ifdim\Gin@nat@width>\linewidth\linewidth\else\Gin@nat@width\fi}
\def\maxheight{\ifdim\Gin@nat@height>\textheight\textheight\else\Gin@nat@height\fi}
\makeatother
% Scale images if necessary, so that they will not overflow the page
% margins by default, and it is still possible to overwrite the defaults
% using explicit options in \includegraphics[width, height, ...]{}
\setkeys{Gin}{width=\maxwidth,height=\maxheight,keepaspectratio}
% Set default figure placement to htbp
\makeatletter
\def\fps@figure{htbp}
\makeatother
\setlength{\emergencystretch}{3em} % prevent overfull lines
\providecommand{\tightlist}{%
  \setlength{\itemsep}{0pt}\setlength{\parskip}{0pt}}
\setcounter{secnumdepth}{5}

\title{Long-term trends in cadmium and lead concentrations in \emph{Mytilus
edulis} samples from European Atlantic Coast}
\usepackage{etoolbox}
\makeatletter
\providecommand{\subtitle}[1]{% add subtitle to \maketitle
  \apptocmd{\@title}{\par {\large #1 \par}}{}{}
}
\makeatother
\subtitle{\url{https://github.com/slm119/ENV872_FinalProject}}
\author{Sena McCrory}
\date{}

\begin{document}
\maketitle

\newpage
\tableofcontents 
\newpage
\listoftables 
\newpage
\listoffigures 
\newpage

\hypertarget{rationale-and-research-questions}{%
\section{Rationale and Research
Questions}\label{rationale-and-research-questions}}

Heavy metal pollution is a global environmental and human health
concern. Cadmium and lead are two heavy metals which continue to be a
major public health concern (according to the World Health
Organization). Increasingly, research indicates that there may in fact
be no ``safe'' level of cadmium or lead exposure, and so reducing levels
of these neurotoxic, carcinogenic metals is a continuing battle for
global health. Cadmium and lead are both naturally occurring in the
environment, but their concentrations have been significantly increased
due to human activity including fossil fuel combustion, industrial
manufacturing, mining activities, among others. Monitoring the spatial
and temporal distribution of these heavy metal pollutants is essential
to tracking trends and identifying pollution hot spots.

Concentrations of heavy metals in the soft tissues of bivalves like
\emph{Mytilus edulis} (blue mussel) have been used for decades as a
bioindicator for heavy metal pollution in marine environments because
these bivalves are capable of bioaccumulating heavy metals from their
environment. \emph{Mytilus edulis} are also harvested and farmed
commercially for human consumption, and so monitoring heavy metals in
their tissues is also important for managing potential human health
risks from consumption. The International Council for the Exploration of
the Sea (ICES) maintains an extensive database of contaminants in marine
biota which includes sampling data for \emph{Mytilus edulis} from
several long-term sampling programs throughout Europe dating back to
1979. Median concentrations of cadmium and lead have decreased
noticeable between 1979 and 1991 likely due to environmental regulations
put in place during that time period such as the banned use of
tetraethyl leaded gasoline; however, this decreased has slowed
noticeably in recent decades (Fig 1).

For this analysis, I explored two questions concerning cadmium and lead
concentrations in \emph{Mytilus edulis} using the ICES monitoring data
from 1990 to 2018.

\begin{enumerate}
\def\labelenumi{\arabic{enumi}.}
\tightlist
\item
  How have concentrations of cadmium and lead in \emph{Mytilus edulis}
  changed overall in the study region since the 1990s?
\item
  Do cadmium and lead concentrations in \emph{Mytilus edulis} differ by
  country?
\end{enumerate}

\newpage

\hypertarget{dataset-information}{%
\section{Dataset Information}\label{dataset-information}}

The global coastline data used in the study area map is publicly
available through the National Oceanographic and Atmospheric
Administration's (NOAA) National Center for Environmental Information
(NCEI) online data portal located here:
\url{https://www.ngdc.noaa.gov/mgg/shorelines/}. The L1 resolution files
were used from the Global Self-consistent Hierarchical High-resolution
Geography (GSHHG) dataset. Geographic reference system is WGS84 (decimal
degrees).\\
Data for cadmium and lead concentrations in \emph{Mytilus edulis} were
downloaded from the International Council for the Exploration of the Sea
(ICES) DOME database on Feb 26, 2020 (available here:
\url{http://dome.ices.dk/views/ContaminantsBiota.aspx}). This data
portal holds a collection of marine related monitoring data sourced from
several regional European monitoring groups including ICES, OSPAR,
HELCOM, AMAP, and Expert Groups. Data for all metal and metalloid
concentrations in biota were downloaded and then filtered to include
only \emph{Mytilus edulis} species and the specific heavy metals of
interest for this study. The sampling data for \emph{Mytilus edulis} was
restricted to include on concentrations reported for the ``whole soft
body'' of the bivalve and expressed in mass of metal per mass of the
organism wet weight. There was no information provided to allow a
conversion from dry weight to wet weight concentrations, and so dry
weight records were excluded from the analysis. Additionally, records
flagged as having ``suspect'' data quality were excluded from the data
set. Dataset variables are described in Table 1 below.

\begin{longtable}[]{@{}llll@{}}
\caption{Description and statistics for variables in ICES DOME
monitoring data for cadmium and lead in \emph{Mytilus
edulis}}\tabularnewline
\toprule
\begin{minipage}[b]{0.15\columnwidth}\raggedright
Variable name\strut
\end{minipage} & \begin{minipage}[b]{0.22\columnwidth}\raggedright
Description\strut
\end{minipage} & \begin{minipage}[b]{0.26\columnwidth}\raggedright
Statistics for Cd\strut
\end{minipage} & \begin{minipage}[b]{0.26\columnwidth}\raggedright
Statistics for Pb\strut
\end{minipage}\tabularnewline
\midrule
\endfirsthead
\toprule
\begin{minipage}[b]{0.15\columnwidth}\raggedright
Variable name\strut
\end{minipage} & \begin{minipage}[b]{0.22\columnwidth}\raggedright
Description\strut
\end{minipage} & \begin{minipage}[b]{0.26\columnwidth}\raggedright
Statistics for Cd\strut
\end{minipage} & \begin{minipage}[b]{0.26\columnwidth}\raggedright
Statistics for Pb\strut
\end{minipage}\tabularnewline
\midrule
\endhead
\begin{minipage}[t]{0.15\columnwidth}\raggedright
PARAM\strut
\end{minipage} & \begin{minipage}[t]{0.22\columnwidth}\raggedright
Parameter\strut
\end{minipage} & \begin{minipage}[t]{0.26\columnwidth}\raggedright
Cd -- cadmium (6762 records)\strut
\end{minipage} & \begin{minipage}[t]{0.26\columnwidth}\raggedright
Pb -- lead (6709 records)\strut
\end{minipage}\tabularnewline
\begin{minipage}[t]{0.15\columnwidth}\raggedright
MYEAR\strut
\end{minipage} & \begin{minipage}[t]{0.22\columnwidth}\raggedright
Monitoring year\strut
\end{minipage} & \begin{minipage}[t]{0.26\columnwidth}\raggedright
1990 - 2018\strut
\end{minipage} & \begin{minipage}[t]{0.26\columnwidth}\raggedright
1990 - 2018\strut
\end{minipage}\tabularnewline
\begin{minipage}[t]{0.15\columnwidth}\raggedright
DATE\strut
\end{minipage} & \begin{minipage}[t]{0.22\columnwidth}\raggedright
Sample date\strut
\end{minipage} & \begin{minipage}[t]{0.26\columnwidth}\raggedright
February 6, 1990 to Feb 27, 2019\strut
\end{minipage} & \begin{minipage}[t]{0.26\columnwidth}\raggedright
February 6, 1990 to Feb 27, 2019\strut
\end{minipage}\tabularnewline
\begin{minipage}[t]{0.15\columnwidth}\raggedright
Latitude and Longitude\strut
\end{minipage} & \begin{minipage}[t]{0.22\columnwidth}\raggedright
units: decimal degrees\strut
\end{minipage} & \begin{minipage}[t]{0.26\columnwidth}\raggedright
\strut
\end{minipage} & \begin{minipage}[t]{0.26\columnwidth}\raggedright
\strut
\end{minipage}\tabularnewline
\begin{minipage}[t]{0.15\columnwidth}\raggedright
Country\strut
\end{minipage} & \begin{minipage}[t]{0.22\columnwidth}\raggedright
Country where measurement was reported\strut
\end{minipage} & \begin{minipage}[t]{0.26\columnwidth}\raggedright
\strut
\end{minipage} & \begin{minipage}[t]{0.26\columnwidth}\raggedright
\strut
\end{minipage}\tabularnewline
\begin{minipage}[t]{0.15\columnwidth}\raggedright
Value.mgperkg\strut
\end{minipage} & \begin{minipage}[t]{0.22\columnwidth}\raggedright
Concentration of contaminant in subsample. Units: mg metal/kg organism
mass (wet weight of whole soft body).\strut
\end{minipage} & \begin{minipage}[t]{0.26\columnwidth}\raggedright
Cd: range DL -- 38.9; median 0.18; mean 0.32\strut
\end{minipage} & \begin{minipage}[t]{0.26\columnwidth}\raggedright
Pb: range DL - 98.01; median 0.28; mean 0.01\strut
\end{minipage}\tabularnewline
\begin{minipage}[t]{0.15\columnwidth}\raggedright
NOINP\strut
\end{minipage} & \begin{minipage}[t]{0.22\columnwidth}\raggedright
Number of individuals included in the subsample\strut
\end{minipage} & \begin{minipage}[t]{0.26\columnwidth}\raggedright
range 1-703\strut
\end{minipage} & \begin{minipage}[t]{0.26\columnwidth}\raggedright
range 1-703\strut
\end{minipage}\tabularnewline
\begin{minipage}[t]{0.15\columnwidth}\raggedright
DETLI.mgperkg\strut
\end{minipage} & \begin{minipage}[t]{0.22\columnwidth}\raggedright
Reported detection limit of measurement equipment, units in mg/kg\strut
\end{minipage} & \begin{minipage}[t]{0.26\columnwidth}\raggedright
range 0.000007 to 1; 1792 unreported\strut
\end{minipage} & \begin{minipage}[t]{0.26\columnwidth}\raggedright
range 0.0001 - 0.6; 1736 unreported\strut
\end{minipage}\tabularnewline
\begin{minipage}[t]{0.15\columnwidth}\raggedright
QFLAG\strut
\end{minipage} & \begin{minipage}[t]{0.22\columnwidth}\raggedright
Quality flag (see DOME metadata for full description of codes)\strut
\end{minipage} & \begin{minipage}[t]{0.26\columnwidth}\raggedright
24 \textless; 1 D; 2 Q; 6735 NA\strut
\end{minipage} & \begin{minipage}[t]{0.26\columnwidth}\raggedright
69 \textless; 0 D; 5 Q; 6635 NA\strut
\end{minipage}\tabularnewline
\bottomrule
\end{longtable}

\newpage

\hypertarget{exploratory-analysis}{%
\section{Exploratory Analysis}\label{exploratory-analysis}}

\hypertarget{geographic-distribution-of-samples}{%
\subsection{Geographic distribution of
samples}\label{geographic-distribution-of-samples}}

\begin{figure}
\includegraphics[width=0.9\linewidth]{C:/Users/senam/Box Sync/My Documents/MEM classes/Duke Spring 2020/DataAnalytics/ENV872_FinalProject/Output/StudyRegionMap} \caption{Sample locations from 1990 to 2018}\label{fig:unnamed-chunk-1}
\end{figure}

Sample locations were highly clustered near European Atlantic and North
Sea shorelines and were similar for cadmium and lead (Fig --). The total
number of samples during the study period varied significantly by
country. For both metals, Norway alone accounted for about 43 percent of
all the records (Fig ---). The top four countries with the most records
(Norway, UK, Ireland, and Denmark) accounted for 90.9 percent of all
records for cadmium and 89.3 percent of all records for lead. In
general, sample numbers were similar between cadmium and lead because in
many cases each \emph{Mytilus edulis} sample was tested for both types
of metals except Spain did not include many records for cadmium.
Sampling records were sparse for coastlines in the Baltic Sea. Countries
which did not include any sampling records in this database include
France, Sweden, Finland, Estonia, Lithuania, and Poland.

\begin{figure}
\centering
\includegraphics{McCrory_ENV972_Project_files/figure-latex/unnamed-chunk-2-1.pdf}
\caption{Number of cadmium and lead monitoring records (\emph{Mytilus
edulis}) by country between 1990 and 2019.}
\end{figure}

\hypertarget{temporal-distribution-of-sample-data}{%
\subsection{Temporal distribution of sample
data}\label{temporal-distribution-of-sample-data}}

Yearly median concentrations in \emph{Mytilus edulis} for both cadmium
and lead concentrations show a noticeable decrease between 1979 to about
1990 (Fig --). After 1990, however, it is unclear whether there
continues to be a decrease in heavy metal concentration; a monthly time
series analysis is used to determine whether a there is a monotonic
trend after 1990. Additionally, monitoring data was more sparse before
1990 and too much interpolation would be needed to include it in a trend
analysis. For both metals, April to July were most likely to have
missing sample data during the study period and September through
November were the most compete months with no missing monthly data.

\begin{figure}
\centering
\includegraphics{McCrory_ENV972_Project_files/figure-latex/unnamed-chunk-3-1.pdf}
\caption{Yearly median concentrations of cadmium and lead in
\emph{Mytilus edulis} ICES monitoring data from 1979 to 2018.}
\end{figure}

\begin{longtable}[]{@{}rrr@{}}
\caption{Summary of the number times each month was interpolated over
the study period from Feb 1990 to Jan 2019.}\tabularnewline
\toprule
Month & Cd & Pb\tabularnewline
\midrule
\endfirsthead
\toprule
Month & Cd & Pb\tabularnewline
\midrule
\endhead
1 & 13 & 12\tabularnewline
2 & 7 & 8\tabularnewline
3 & 6 & 6\tabularnewline
4 & 16 & 16\tabularnewline
5 & 21 & 21\tabularnewline
6 & 20 & 20\tabularnewline
7 & 20 & 21\tabularnewline
8 & 7 & 7\tabularnewline
9 & 0 & 0\tabularnewline
10 & 0 & 0\tabularnewline
11 & 0 & 0\tabularnewline
12 & 7 & 7\tabularnewline
\bottomrule
\end{longtable}

\hypertarget{distribution-of-cadmium-and-lead-sample-concentrations}{%
\subsection{Distribution of cadmium and lead sample
concentrations}\label{distribution-of-cadmium-and-lead-sample-concentrations}}

Concentrations for cadmium ranged from below the instrument detection
limit to 98.1 mg/kg with a median of 0.28 mg/kg and mean of 0.81 mg/kg.
Lead concentrations ranged from below the detection limit to 38.9 mg/kg
with a median of 0.18 mg/kg and mean of 0.32 mg/kg. Both metals
exhibited a strong positive skew and so are displayed on a log scale in
most figures. Detection limits for measurements instruments ranged
substantially for both metal samples (Table 1), and many samples
concentrations were within the range of detection limits for other
samples (Fig **).

\begin{figure}
\centering
\includegraphics{McCrory_ENV972_Project_files/figure-latex/unnamed-chunk-5-1.pdf}
\caption{Distirbution of sample concentrations of cadmium and lead in
\emph{Mytilus edulis} from ICES data 1990 to 2019.}
\end{figure}

\begin{figure}
\centering
\includegraphics{McCrory_ENV972_Project_files/figure-latex/unnamed-chunk-6-1.pdf}
\caption{Cadmium and lead sample concentrations over time for ICES
\emph{Mytilus edulis} monitoring. Grey lines show the various instrument
detection limits reported in each the data set.}
\end{figure}

\newpage

\hypertarget{analysis}{%
\section{Analysis}\label{analysis}}

\hypertarget{question-1-how-have-cadmium-and-lead-concentrations-in-mytilus-edulis-changed-over-time}{%
\subsection{\texorpdfstring{Question 1: How have cadmium and lead
concentrations in \emph{Mytilus edulis} changed over
time?}{Question 1: How have cadmium and lead concentrations in Mytilus edulis changed over time?}}\label{question-1-how-have-cadmium-and-lead-concentrations-in-mytilus-edulis-changed-over-time}}

Data for lead and cadmium concentrations were aggregated using the
monthly median for each month between February 1990 to January 2019.
Months with no sample data were interpolated using a linear
interpolation method (Fig **).

Cadmium concentrations in \emph{Mytilus edulis} have decreased
significantly over the study period by an estimated 0.002 mg/kg from
1990 to 2019 (Seasonal Mann-Kendall; S = -1044; p \textless0.0001,
Seasonal Sen's Slope). Cadmium concentrations also exhibited a seasonal
variation where concentrations in November, December, and January
exhibited significantly different decreasing trend than other months
(Seasonal Mann-Kendall, p \textless{} 0.05). Lead showed no significant
monotonic trend over the study period (Seasonal Mann-Kendall; S = -284;
p = 0.125).

\begin{figure}
\centering
\includegraphics{McCrory_ENV972_Project_files/figure-latex/unnamed-chunk-7-1.pdf}
\caption{Concentrations of cadmium and lead in \emph{Mytilus edulis} in
ICES study region from 1990 to 2018.}
\end{figure}

\hypertarget{question-2-do-cadmium-and-lead-concentrations-in-mytilus-edulis-differ-by-country}{%
\subsection{\texorpdfstring{Question 2: Do cadmium and lead
concentrations in \emph{Mytilus edulis} differ by
country?}{Question 2: Do cadmium and lead concentrations in Mytilus edulis differ by country?}}\label{question-2-do-cadmium-and-lead-concentrations-in-mytilus-edulis-differ-by-country}}

Comparison of concentrations between countries was restricted to only
countries with more than 100 records over the study period. A
non-parametric group-wise comparison was used to compare concentration
distributions by country.

Analyses for cadmium revealed that concentrations distributions differed
significantly between all countries except Germany and the UK which were
statistically similar (Kruskal-Wallis Rank Sum Test, chi-squared =
897.8, df = 5, p \textless{} 0.0001; Dunn's Test with Benjamin-Hochberg
(1995) adjustment, p \textless{} 0.05 for all comparisons except Germany
and UK). The Netherlands had the highest distribution of cadmium
concentrations with a median of 0.41 mg/kg (see Table 3).

Lead concentration distributions between countries also differed
significantly by country (Kruskal-Wallis Rank Sum Test, chi-squared =
1302.8, df = 6, p \textless{} 0.0001; Dunn's Test with Benjamin-Hochberg
(1995) adjustment, alpha = 0.05). The Netherlands, Spain, and UK had the
highest distribution of lead concentrations and were statistically to
each other. Next highest lead concentrations were in Norway, then
Germany and Ireland (not significantly different from each other), and
then the lowest lead concentrations were in Denmark (Dunn's Test, p
\textless{} 0.05).

\begin{figure}
\centering
\includegraphics{McCrory_ENV972_Project_files/figure-latex/unnamed-chunk-8-1.pdf}
\caption{Distribution of cadmium and lead concentrations in
\emph{Mytilus edulis} for countries with more than 100 samples between
1990 and 2018.}
\end{figure}

\begin{longtable}[]{@{}lll@{}}
\caption{Summary of median concentrations of cadmium and lead in
\emph{Mytilus edulis} by country during the study period from 1990 --
2019.}\tabularnewline
\toprule
Country & Median Cd conc (mg/kg) & Median Pb conc (mg/kg)\tabularnewline
\midrule
\endfirsthead
\toprule
Country & Median Cd conc (mg/kg) & Median Pb conc (mg/kg)\tabularnewline
\midrule
\endhead
The Netherlands & 0.41 & 0.55\tabularnewline
Spain & NA & 0.49\tabularnewline
Norway & 0.21 & 0.25\tabularnewline
Germany & 0.2 & 0.24\tabularnewline
United Kingdom & 0.191 & 0.54\tabularnewline
Denmark & 0.164 & 0.15\tabularnewline
Ireland & 0.11 & 0.22\tabularnewline
\bottomrule
\end{longtable}

\newpage

\hypertarget{summary-and-conclusions}{%
\section{Summary and Conclusions}\label{summary-and-conclusions}}

\newpage

\hypertarget{references}{%
\section{References}\label{references}}

\textless add references here if relevant, otherwise delete this
section\textgreater{}

\end{document}
